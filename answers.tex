\documentclass{article}
\usepackage{graphicx} % Required for inserting images

\usepackage{fontspec}
\usepackage{polyglossia}
\usepackage{xcolor}

\setdefaultlanguage{ukrainian}
\setotherlanguages{english}

\newfontfamily\cyrillicfont{Times New Roman}[Script=Cyrillic]
\setmainfont[Script=Cyrillic]{Times New Roman}

\usepackage{geometry}

\geometry {
    a4paper,
    left=10mm,
    top=10mm,
    right=10mm,
    bottom=15mm
}

\title{Екзамен. Крипта}
\author{Nikita Isachenko}
\date{June 2024}

\begin{document}

\maketitle

\vspace{2cm}

\tableofcontents

\pagebreak

\section{Цілі, напрямки, методи і аспекти захисту інформації. Криптологія. Задачі криптографії та криптоаналізу. Початкові поняття криптології та етапи розвитку. Класифікація криптосистем.}

Тут має бути якийсь текст

\section{Класична криптографія: терміни, поняття, позначення, типи шифрів. Визначення шифру підстановки (заміни). Моноалфавітні підстановки: визначення, загальний шифр простої підстановки. Моноалфавітні шифри класичної криптографії: Цезаря, афінної заміни, шифр Полібія та інші. Частотний аналіз шифру Цезаря та афінної підстановки.}

Тут має бути якийсь текст

\section{Блокові (табличні) підстановки: шифр Плейфера, афінна n-грамна заміна, шифр Хілла. Можливості частотного аналізу цих шифрів.}

Тут має бути якийсь текст

\section{Визначення поліалфавітної підстановки. Модульне шифрування. Класичні поліалфавітні шифри: Віженера, багатоконтурний шифр Віженера, шифр з автоключем, аперіодичні поліалфавітні шифри, книжковий шифр з бігучим рядком, шифр Вернама (одноразовий блокнот). Частотний аналіз шифру Віженера. Поняття адитивних або шифрів модульного гамування.}

Тут має бути якийсь текст

\section{Класична криптографія. Визначення шифру загальної блокової перестановки. Класичні шифри перестановки: Скітала, частоколу, табличні перестановки, маршрути Гамільтона, грати Кардано, магічні квадрати, n-кратні перестановки. Класифікація класичних шифрів.}

Тут має бути якийсь текст

\section{Поняття ентропії, властивості ентропії імовірнісних ансамблів, сумісна та умовна ентропія, взаємна інформація.}

Тут має бути якийсь текст

\section{Джерела дискретних сигналів, ентропія на символ джерела, надлишковість. Моделі джерел відкритого тексту.}

Тут має бути якийсь текст

\section{Поняття стійкості, теоретична і практична стійкість. Правило Керкгоффса. Ієрархія типів атак на криптосистему за рівнем доступної криптоаналітику інформації. Відмінність в криптоаналізі на основі шифрованих текстів та на основі відкритих текстів для класичних шифрів. Загальна схема секретного зв'язку.}

Тут має бути якийсь текст

\section{Поняття криптосистеми. Математична модель Шеннона симетричного шифру. Припущення Шеннона. Формули для розрахунку сумісних і умовних розподілів в математичній моделі шифру.}

Тут має бути якийсь текст

\section{Цілком таємна криптосистема. Необхідні і достатні умови цілковита таємності. Межа Шеннона. Цілковита таємність шифру Вернама.}

Тут має бути якийсь текст

\section{Ненадійність ключа і відкритого тексту. Теореми про ентропією ключів за умовою криптограми та про середнє число хибних ключів (із доведенням).}

Тут має бути якийсь текст

\section{Функція ненадійності ключа. Відстань однозначності: визначення, доведення формули, інтерпретація, застосування. Принципи Шеннона: розсіювання і перемішування. Підхід до побудови стійких криптосистем, запропонований Шенноном. Класифікація сучасних криптосистем.}

Тут має бути якийсь текст

\section{Випадкові та псевдовипадкові послідовності в криптографії. Вимоги до випадкових послідовностей в криптографії.}

Тут має бути якийсь текст

\section{Формальні підходи до визначення поняття випадкової послідовності}

Тут має бути якийсь текст

\section{Первинні джерела випадкових шумів. Способи перетворення первинного шуму в випадкові дискретні послідовності. Математичні перетворення для покращення якості випадкових послідовностей.}

Тут має бути якийсь текст

\section{Одновимірні та багатовимірні булеві функції. Способи представлення булевих функцій: таблиці істинності, формули, ДДНФ, розклад Шеннона.}

Тут має бути якийсь текст

\section{Поліном Жегалкіна (АНФ), алгебраїчний степінь булевої функції. Швидке перетворення Мебіуса.}

Тут має бути якийсь текст

\section{Спектральні представлення булевих функцій. Ряд та коефіцієнти Фур’є, перетворення та коефіцієнти Уолша.}

Тут має бути якийсь текст

\section{Швидке перетворення Фур’є. Властивості коефіцієнтів Фур’є та Уолша, рівність Парсеваля.}

Тут має бути якийсь текст

\section{Криптографічні властивості булевих функцій. Невиродженість, відсутність заборон, збалансованість, згладжування.}

Тут має бути якийсь текст

\section{Статистичні аналоги булевих функцій. Нелінійність як відстань до класу афінних функцій, вивід формули, оцінка. Поняття бент-функції.}

Тут має бути якийсь текст

\section{Кореляційний імунітет булевих функцій: різні визначення, зв’язок із коефіцієнтами Уолша (із доведенням).}

Тут має бути якийсь текст

\section{Лавинні ефекти булевих функцій. Строгі лавинні критерії та критерії поширення. Похідні булевих функцій, функція автокореляції та її зв’язок із критеріями поширення.}

Тут має бути якийсь текст

\section{Симетричні блокові шифри: визначення, загальні властивості. Принципи побудови сучасних блокових шифрів. Модель ітеративного шифру.}

Тут має бути якийсь текст

\section{Схеми блокового шифрування: SP-мережа, схема Фейстеля, їх властивості.}

Тут має бути якийсь текст

\section{Стандарт шифрування DES: схема роботи, характеристики, недоліки. Модифікації алгоритму DES.}

Тут має бути якийсь текст

\section{Стандарт шифрування ДСТУ ГОСТ 28147:2009: схема роботи, характеристики.}

Тут має бути якийсь текст

\section{Стандарт шифрування AES: схема роботи, структура, характеристики. Швидка реалізація AES.}

Тут має бути якийсь текст

\section{Стандарти шифрування ДСТУ 7624:2014 «Калина» та ГОСТ Р 34.12-2015 «Кузнєчік»: схема роботи, основні характеристики.}

Тут має бути якийсь текст

\section{Режими роботи блокових шифрів, основні характеристики. Вплив спотворень у шифротекстах на відкриті тексти у різних режимах роботи.}

Тут має бути якийсь текст

\section{Потокові шифри: визначення, загальна модель. Типи генераторів гами. Внесення нелінійності у схеми на основі регістрів зсуву із лінійним зворотним зв’язком.}

Тут має бути якийсь текст

\section{Типи атак на потокові шифри. Кореляційна атака на схему нелінійної комбінації (на прикладі генератору Джиффі).}

Тут має бути якийсь текст

\section{Потокові шифри А5/1, А5/2. Потоковий шифр RC4. Конкурс eSTREAM.}

Тут має бути якийсь текст

\end{document}
